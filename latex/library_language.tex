\section{Library Language}
The Domain Specific Language we have created is an attempt at modelling a Library. We have a chosen the context of a public library as opposed to a specialised facility such as a university or law library. The DSL is naturally constrained by our development tool and as such  is comprised of three main entities. Objects represent physical entities such as loanable items, library equipment and people such as library users and employees. Relationships describe methods by which objects within the DSL may interact. Roles define constraints on the types of objects that may mutually INHABIT? TAKE THE PLACE IN THE RELATIONSHIP, ?BE?, ?RESIDE? word please  end of a relationship between those objects. Additionally constraints may be defined surrounding the semantics of relationships. These constraints fall into four categories:
Connectivity: cardinality and type of some shit about the thing with a thing. 
Occurrence: constraint surrounding the number of a unique objects may appear in a graph
Uniqueness: uniqueness constraints for property values such as ID etc
Port: No fucking idea what the shit port was for, probably something to do with ships.
Creating and applying constraints allows further development and enrichment of DSL semantics.  
\subsection{Objects}
\includegraphics[width=0.5\textwidth]{obj_book}
\includegraphics[width=0.5\textwidth]{obj_author}
\includegraphics[width=0.5\textwidth]{obj_librarian}
\includegraphics[width=0.5\textwidth]{obj_patron}

\subsection{Relationships/Roles}
\includegraphics[width=0.5\textwidth]{rel_wrote}
\includegraphics[width=0.5\textwidth]{rel_reserve}
\includegraphics[width=0.5\textwidth]{rel_records}