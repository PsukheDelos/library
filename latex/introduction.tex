\section{Introduction}
The term Domain-Specific Language (DSL) is heard a lot nowadays. A DSL is a language developed to address the need of a given domain. This domain can be a problem domain (e.g. insurance, healthcare, transportation) or a system aspect (e.g. data, presentation, business logic, workflow). The idea is to have a language with limited concepts which are all focused on a specific domain. This leads to higher level languages improving developer productivity and communication with domain experts. In a lot of cases it is even possible to let domain experts use the DSL and develop applications.

In this project we create a simple library management system in MetaEdit+. The solution consists of some basic building blocks, such as books, authors, librarians. 