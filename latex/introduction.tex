\section{Introduction}
The term Domain-Specific Language (DSL) is heard a lot nowadays. A DSL is a language developed to address the need of a given domain. This domain can be a problem domain (e.g. insurance, healthcare, transportation) or a system aspect (e.g. data, presentation, business logic, work flow).
The idea is to have a language with limited concepts which are all focused on a specific domain and describe a meta model therein~\cite{karsai2014design}.
This allows teams to focus development efforts on accurately and effectively describing the domain space as opposed to being caught up in low level implementation details.\par
Enhanced focus leads to higher level expressions of the model improving developer productivity and communication with domain experts. The inherent complexity and propensity for error with requirements specifications and analysis is reduced.
In a lot of cases it is even possible to let domain experts use the DSL and develop applications. The combination of these elements leads to more flexible and robust applications that better serve their intended user base.

In this project we created a Domain Specific Language for a simple library management system. We used a meta case modelling tool called MetaEdit+.  The solution consists of some basic building blocks, such as books, authors, librarians. The system describes the appropriate relationships between these entities and the required constraints to ensure the system functions as expected and intended.