\section{Language Design Background}
Despite the lengthy history and recent popularity of domain-specific languages, the task of actually
designing DSLs remains a difficult and under-explored problem.

People find DSLs valuable because a well-designed DSL can be much easier to program with than a traditional library. This improves programmer productivity, which is always valuable. In particular it may also improve communication with domain experts, which is an important tool for tackling one of the hardest problems in software development.
Traditionally, the definition of a language proceeds from syntax to semantics. That is, first a syntax is
defined, then a semantic model is decided upon, and finally the syntax is related to the semantic model. 

\subsection{Benefits of DSLs}
Using DSLs can reap a multitude of benefits. 

Once you’ve got a language and its execution engine for a particular aspect of your development task, work becomes much more efficient, simply because you don’t have to do the grunt work manually. 

Using DSLs can increase the quality of the created product: fewer bugs, better architectural conformance, increased maintainability.

Since DSLs capture their respective concern in a way that is not cluttered with implementation details, DSL programs are more semantically rich than GPL programs.

DSLs whose domain, abstractions and notations are closely aligned with how domain experts (i.e. non-programmers), this allows for very good integration between developers and domain experts: domain experts can easily read, and often write program code, since it is not cluttered with implementation details irrelevant to them. 