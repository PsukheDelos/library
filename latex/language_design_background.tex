\section{Language Design Background}
Despite the lengthy history and recent popularity of domain-specific languages, the task of actually
designing DSLs remains a difficult and under-explored problem.

People find DSLs valuable because a well-designed DSL can be much easier to program with than a traditional library. This improves programmer productivity, which is always valuable. In particular it may also improve communication with domain experts, which is an important tool for tackling one of the hardest problems in software development.
Traditionally, the definition of a language proceeds from syntax to semantics. That is, first a syntax is
defined, then a semantic model is decided upon, and finally the syntax is related to the semantic model. 